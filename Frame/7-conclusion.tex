
This shows how an implementation of $\lamto$ and its interpretation into a CCC
can be formalized in Agda.

\medskip

Nevertheless, some aspects of the code are not entirely satisfactory:
\begin{itemize}
\item
The definitions of finite types and finite lists are problematic. Their encoding
was chosen to fit the problems encountered early on. But in chapter
\ref{ch:interpretation}, this resulted in many correction arrows having to be
added, making theorems and proofs needlessly more complex. Also, because of how
these arrows are defined, the interpretation function for judgements does not
compute.

These problems could have been avoided with a recursive definition of
lists - but this in turn would produce its own set of problems, presumably
around the definition of weakening, since there, arbitrary indices into a list
have to be dealt with.

Such a conflict arises because a single definition is used everywhere, and it
could be solvable if different definitions which fundamentally describe the
same object could be used interchangebly.

Fortunately, this seems to be exactly what the concept of univalence provides and it would
be interesting to see how it can be applied here.

\item
  The code was only written with focus on correctness, not on performance.
  Because of this, evaluating the $β$-equality of simple terms may
  sometimes take multiple minutes, while using several gigabytes of memory.

  Since this evaluation is done as part of typechecking a file, using the CTT
  normalization of Agda, it is not immediately clear what the reason for
  such performance problems could be.

  Still, before any practical usage can happen, this direction has to be
  explored as well.
\end{itemize}

\medskip

Concerning the goal of visualizing programs, this exploration of the
semantics of $\lamto$ reinforces the idea that category theory and diagrams
should be useful tools: it is natural to think about computations as morphisms
from a context to their result type.

But it also raises questions: Arrows have
no clear representation. As encoded by the very concept of a commutating
diagram, the same computation could be visualized using different combinations
of arrows. This indicates that there is no canonical way to represent a program
diagrammatically, while at the same time, it could be possible to leverage this
fact for zooming in and out of a representation.

\medskip

Finally, $\lamto$ is a very simple type theory, missing many of the
constructions found in real world programming languages (sum types, general data
types, dependent types). Because of this, the next natural step would be to
extend it with these concepts, while also moving to appropriate categories,
where the corresponding semantics can be formulated.
