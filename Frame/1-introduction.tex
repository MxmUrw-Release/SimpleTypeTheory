

Type theory was first introduced by Russell in 1903 \citep{kamareddine2002} as
part of Russell and Whitehead's efforts of writing the \textit{Principia
  Mathematica}, where its role was to serve as a safe, but still powerful basis for formalizing mathematics -
escaping the paradoxes of set theory, such as, ``Does the set of all sets, which do not
contain themselves, contain itself?'' (Russell's paradox). \citep{ConstableManuscript}

\medskip

It is closely related to intuitionistic (constructive) mathematics, a branch of
mathematics where proofs are meant to be constructive. Here, a proof of the existence
of some object must give us a way to explicitly construct such an object. In
practice, this means that axioms which allow us to circumvent explicit
constructions, like the principle of excluded middle, are rejected. In the early
1930s, Brouwer, Heyting and Kolmogorov gave a computational interpretation of
intuitionistic logic. The discovered principle is known as ``Propositions as
Types'' or ``Curry-Howard Isomorphism'', the idea is to interpret a proposition
as a type, and a proof of it as an algorithm, stated in the lambda calculus, having this type.

Considering the computational aspect as being crucial, Bishop developed analysis in a
constructive setting (1967). Inspired by this, different type theories were
developed to provide a system in which Bishop's mathematics could be
formalized. These type theories form the basis of modern proof assistants,
including Agda (Intuitionistic Type Theory, \citep{ITT}) and Coq
(Calculus of Inductive Constructions, \citep{CIC}). \citep{ConstableManuscript}

\medskip

In 1945, Eilenberg and MacLane introduced category theory, first as a tool for
applying algebraic methods to a topological problem. Over time, it developed
into a language which could be used to describe the objects of many different
branches of mathematics and faciliated the discovery of connections between
them. Starting in the 1960s, Lawvere and others explored the idea of applying
category theory to the basis of math itself: logic and set theory.
This endeavour resulted in the definition of a topos, a special kind of
category, which has an internal logic rich enough to serve as a generalized
foundational framework. \citep{LandryMarquis}

\medskip

Furthermore, a direct connection to type theory became apparent, where
categories can be seen as providing the semantics for type theories.
Thus summarizing, the relationships between mathematics, type theory and category theory can be
stated as follows \citep{Trinity}:

\begin{center}
\begin{tikzcd}[math mode=false, row sep=huge, column sep=tiny]
  &
  mathematics
    \ar[dl, swap, "\textit{can be formalized in a}"]
    \ar[dr, "\textit{can be internalized in a}"]
    &
    \\
    \begin{tabular}{c}
      type theory \\
      (syntax)
    \end{tabular}
    \ar[rr, swap, align=center, "\textit{can be} \\ \textit{interpreted in a}"]
    &
    &
    \begin{tabular}{c}
      category \\
      (semantics)
    \end{tabular}
\end{tikzcd}
\end{center}

As part of research by Awodey, Warren and Voevodsky around 2006, Homotopy Type
Theory (HoTT) was developed. In homotopy theory (without \textit{-Type-}), topological spaces are studied
with respect to what paths can be constructed. This includes an infinite
hierarchy of paths: between points, between paths between points, between paths
between paths between points, and so on. Thus HoTT, being a type theory which has an
interpretation in Kan simplicial sets (a category studied in homotopy
theory), mirrors these features and offers new ways for doing mathematics in
it \citep{HoTTBook}:
\begin{enumerate}
\item A proof of equality $a = b$ is interpreted as a path between $a$ and $b$.
  In homotopical fashion, such a proof is thus no longer unique: there may be
  different paths between $a$ and $b$. And since such paths may be compared
  again and again, an infinite path structure emerges.
\item HoTT contains a new axiom: the univalence axiom. It says that 
  isomorphic structures may be treated as being equal:
  \[
    (A \simeq B) \simeq (A = B)
  \]
  Such a statement is not consistent with set theory, but it can be assumed in HoTT,
  allowing us to treat isomorphic structures more intuitively. 
\end{enumerate}

\medskip

Nevertheless, the univalence axiom is only an axiom in HoTT, i.e., it has no
computational meaning - code which uses it cannot be executed. This lead
to the development of Cubical Type Theory (CTT), which, by modelling equality \textit{explicitly}
by paths, succeeded in giving computational meaning to the univalence axiom
\citep{CubicalTT}.

\medskip

While the development is still in progress (for example, CTT currently has no
interpretation in as general a class of categories as HoTT has \citep{ShulmanCTT}),
the state of research can be visualized as follows \citep{Trinity}:

\begin{center}
\begin{tikzcd}[math mode=false, row sep=huge, column sep=tiny]
                     & homotopical mathematics \ar[dl, swap, "\textit{can be formalized
                       in}"] \ar[dr, "\textit{can be internalized in}"] & \\
  HoTT / CTT \ar[rr, swap, align=center, "\textit{can be} \\
  \textit{interpreted in}"] & & higher categories
\end{tikzcd}
\end{center}

\medskip

The goal of this thesis is the exploration of the relationship between type
theory and category theory, albeit on a much smaller scale: Our topic is the
simply typed lambda calculus and its interpretation into a cartesian closed
category (CCC):

\begin{center}
\begin{tikzcd}[math mode=false, row sep=huge, column sep=tiny]
                     & (Propositional Logic) \ar[dl, dashed] \ar[dr, dashed] & \\
  ST. $λ$-Calc \ar[rr, swap, align=center, "\textit{can be} \\
  \textit{interpreted in a}"] & & CCC
\end{tikzcd}
\end{center}

In order to present this connection, we formalize it in Cubical Type Theory, as
implemented in the Agda proof assistant. This allows us to take advantage of a
computing functional extensionality (a corrollary of univalence, does not
compute in other type theories). Because of this, some practical aspects of formalizing
mathematics in a (cubical) type theory are also touched upon in this thesis.

\section*{Motivation}
My motivation comes from the praxis of software development. Sometimes, when
programming, I would like to have a tool which could provide a higher level
view on code: To see the whole
structure of a program at once, to see which components exist and how they
interact. Especially, this would make it easier to explore and navigate large codebases.

But it should be more than a mere visualization of the code. It should be an
interactive representation equivalent to it, such that it would
become possible to program on a higher level - graphically managing the
connections between lower level components. Also it should be possible to zoom
in and out arbitrarily, enabling the programmer to work on different abstraction
levels.

\medskip

I would like to work towards this vision, and the deep connection between
programming (type theory) and category theory seems to be a promising tool.
Particularly, because category theory makes extensive use of visualizations (in
the form of diagrams), while also being a natural framework for repeated
abstractions.

Furthermore, realizing this vision would involve writing an interpreter, and in
order to eliminate errors, it should be written in a language with a strong typing
system. 

\medskip

Therefore, I take this thesis as an opportunity to combine both aspects: By
formalizing in Agda, I can implement an interpreter and formally verify its
correctness, and then continue by exploring the connection between programming
and category theory.


\section*{Structure}
The following chapters are structured as follows: In chapter 2, Agda and its
syntax are introduced as an example of working in a type theory. In chapter 3,
some types which occur frequently are presented. In chapter 4, category theory
is introduced and is developed far enough for the definition of a cartesian
closed category. In chapter 5, the simply typed lambda calculus is introduced.
This involves the definition of a typechecker, context weakening and
substitution, as well as proofs about their behaviour. Furthermore,
$β$-reduction is defined, and normalization of well typed terms is proven.
Finally, in chapter 6, the interpretation of well typed terms in a CCC is given
and proven to be sound with respect to $β$-reduction.













